


\documentclass{beamer}

% Ne dodajajte nastavitve za velikost pisave, kot je bila v datoteki `5-prosojnice.tex`.

\usepackage{predavanja}
\usepackage{bookmark}
\usepackage[T1]{fontenc}
\usepackage[utf8]{inputenc}
\usepackage[slovene]{babel}
\usepackage{dsfont}

\begin{document}




\title{Matematični izrazi in uporaba paketa \texttt{beamer}}
\subtitle{\emph{Matematičnih} nalog ni treba reševati!}
\institute{Fakulteta za matematiko in fiziko}
\date{}

\frame{\titlepage}

% Zgornje podatke nastavite z ukazi kot v dokumentih razreda `article`.
% Več o tem, kako se naredi naslovno stran, si preberite na naslovu na naslovu:
% https://www.overleaf.com/learn/latex/Beamer
% To stran preberite do vključno razdelka "Creating a table of contents".
% Ukaz `\titlepage` deluje podobno kot ukaz `\maketitle`, ki ste ga že srečali.


\newtheorem{definicija}{Definicija}[section]

\newtheorem{izrek}{Izrek}[section]

\begin{frame}


\frametitle{Kratek pregled}
\tableofcontents
\end{frame}



\section{Paket \texttt{beamer}}

\input{prosojnice/1-paket-beamer.tex}

\section{Paketa \texttt{amsmath} in \texttt{amsfonts}}

\input{prosojnice/2-paketa-amsmath-amsfonts.tex}


\section[Matematika, 1. del\\\large{Analiza, logika, množice}]{Matematika, 1. del}

\input{prosojnice/3-analiza-logika-mnozice.tex}

\section{Stolpci in slike}

\section{Paket \texttt{beamer} in tabele}

\section[Matematika, 2. del\\\large{Zaporedja, algebra, grupe}]{Matematika, 2. del}

\end{document}